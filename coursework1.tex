% LaTeX template for Models of Computation assessed coursework
% Most of the required packages are standard and should be provided by most TeX installations
% The exception is mathpartir, which may be downloaded from http://cristal.inria.fr/~remy/latex/

\documentclass[11pt,a4paper]{article}

\usepackage{fullpage}
\usepackage{rotating}
\usepackage{amsmath, amssymb, amsthm}
\usepackage{stmaryrd}
\usepackage{proof}
\usepackage{mathpartir}
\usepackage{tikz}
\usepackage{verbatim}
\usepackage{url}
\usepackage{multicol}
\usepackage{xr-hyper}

% Common macros

% BNF notation
\newcommand{\Gdef}{\mathrel{\mathop{::}}=}
\newcommand{\Gbar}{\mathbin{\ \big|\ }}
\newcommand{\Coloneqq}{\Gdef}

% Big-step arrow
\newcommand{\bigstep}{\mathrel{\Downarrow}}

% Semantic operators are (often) underlined to avoid ambiguity
\newcommand{\semop}[1]{\mathbin{\underline{#1}}}


% Program syntax is set in teletype using the \cmd macro
\newcommand{\cmd}[1]{\texttt{#1}}

% Macros for program constructs
\newcommand{\ifthen}[3]{
  \cmd{if} \; #1 \; \cmd{then} \; #2 \; \cmd{else} \; #3 }
\newcommand{\while}[2]{
  \cmd{while} \; #1 \; \cmd{do} \; #2 }

% \ang{x} typesets x in angled brackets
\newcommand{\ang}[1]{\langle #1 \rangle}

% The following two macros are for typesetting rules and derivations
% Usage: \drule{rule name}{premise1 \\ premise2 \\ premise3 ...}{conclusion}
% The premises will often also be derivations using \drule.
% The difference between \drule and and \Drule is that the space for the rule name
% is not measured with \Drule.  This is useful for typesetting left-most subderivations.
\newcommand{\drule}[3]{\inferrule*[left={#1}]{#2}{#3}}
\newcommand{\Drule}[3]{\inferrule*[Left={#1}]{#2}{#3}}

% For defininitions
\newcommand{\eqdef}{\triangleq}


% WhileDM
\newcommand{\whiledm}{\textsc{WhileDM}}
% Names of types
\newcommand{\tname}[1]{\mathit{#1}}
\newcommand{\exprs}{\tname{Expr}}
\newcommand{\bools}{\tname{Bool}}
\newcommand{\comms}{\tname{Comm}}
\newcommand{\vars}{\tname{Var}}
\newcommand{\nums}{\tname{Num}}
\newcommand{\addrs}{\tname{Addr}}
\newcommand{\vals}{\tname{Val}}
\newcommand{\stor}{\tname{Store}}
\newcommand{\heap}{\tname{Heap}}
\newcommand{\bv}{\tname{BVal}}
\newcommand{\ad}[1]{\ulcorner {#1} \urcorner}
\newcommand{\newp}{\texttt{newpair}}
\newcommand{\fst}[1]{{#1}.\texttt{fst}}
\newcommand{\snd}[1]{{#1}.\texttt{snd}}
\newcommand{\stof}[2]{\texttt{fst} [ {#1} ] \leftarrow {#2}}
\newcommand{\stos}[2]{\texttt{snd} [ {#1} ] \leftarrow {#2}}
\newcommand{\dom}[1]{\mathrm{dom}(#1)}
\newcommand{\bse}{\bigstep_e}
\newcommand{\bsc}{\bigstep_c}
\newcommand{\bsb}{\bigstep_b}
\newcommand{\types}{\tname{Type}}
\newcommand{\typ}{\tau} % Type variable
\newcommand{\tc}{\Gamma} % Type context
\newcommand{\tnat}{\mathsf{nat}}
\newcommand{\tpair}[2]{(#1,#2)}
\newcommand{\hptyp}[3]{#1 \Vdash #2 : #3}
\newcommand{\tcompat}[3]{#1 ; #2 ; #3 \vdash \textsf{\textup{well-typed}}}
\newcommand{\etyp}[3]{#1 \vdash #2 : #3}


\begin{document}
\title{Models of Computation Assessed Coursework I}
\author{Ben Brannick}

\maketitle
\noindent 
\rule{\linewidth}{0.4pt} \\ \\
Question 1:\\
\indent a)
\begin{gather*}
	\drule{e.add}{
		(\dag) \\
		(\ddag) \\
		0 = 0 \semop+ 0
	}{
		\ang{(\fst{\newp}) + (\snd{\newp}), \emptyset, \emptyset} 
		\bse 
		\ang{(0, \emptyset, (1 \mapsto 0, 2 \mapsto 0, 5 \mapsto 0, 6 \mapsto 0)}
	}
	\\ \\
	(\dag) \equiv \drule{e.fst}{
			\drule{e.new}{
				1 \notin \dom h \\ (1 + 1) \notin \dom h
			}{
				\ang{\newp, \emptyset, \emptyset}
				\bse
				\ang{\ad 1, \emptyset, (1 \mapsto 0, 2 \mapsto 0)}			
			}
		}{
			\ang{\fst{\newp}, \emptyset, \emptyset} 
			\bse 
			\ang{0, \emptyset, (1 \mapsto 0, 2 \mapsto 0)}
		}
	\\ \\ 
	(\ddag) \equiv \drule{e.snd}{
			\drule{e.new}{
				5 \notin \dom h \\ (5 + 1) \notin \dom h
			}{
				\ang{\newp, \emptyset, (1 \mapsto 0, 2 \mapsto 0)}
				\bse
				\ang{\ad 5, \emptyset, (1 \mapsto 0, 2 \mapsto 0, 5 \mapsto 0, 6 \mapsto 0)}
			}
		}{
			\ang{\snd{\newp}, \emptyset, (1 \mapsto 0, 2 \mapsto 0)} 
			\bse
			\ang{0, \emptyset, (1 \mapsto 0, 2 \mapsto 0, 5 \mapsto 0, 6 \mapsto 0)}
		}
\end{gather*}
\indent b) \\
\indent The semantics is not deterministic, as there is more than one choice for the value of \emph{a} in \texttt{E.NEW}. This means one program could choose a value of 1 for \emph{a}, whereas another could choose 2. \\
\\
\indent c) \\
\indent The semantics is not total, as not all expressions will evaluate to an answer. For example, the expression \snd{1} is a valid expression, however it requires $\ang{1, s, h} \bse \ang{\ad a, s', h'}$ for some \emph{a}, which is not possible as according to rule \texttt{E.NUM}, $\ang{1, s, h} \bse \ang{1, s, h}$ \\
\\
\indent d) \\
\indent As mentioned above, it is not the case that, for any value \emph{v} or state \emph{s}, $\ang{\snd{1}, s, h} \bse \ang{v, s', h'}$.\\
\clearpage \noindent
\rule{\linewidth}{0.4pt} \\ \\
Question 2: \\
\indent a)
\begin{gather*}
\drule{b.true}{
	\ 
}{
	\ang{\texttt{true}, s, h} \bsb \ang{\texttt{true}, s, h}
} \indent \indent
\drule{b.false}{
	\ 
}{
	\ang{\texttt{false}, s, h} \bsb \ang{\texttt{false}, s, h}
} \\
\drule{b.and.true}{
	\ang{B_1, s, h} \bsb \ang{\texttt{true}, s', h'} \\
	\ang{B_2, s', h'} \bsb \ang{\texttt{true}, s'', h''}
}{
	\ang{B_1 \ \& \ B_2, s, h} \bsb \ang{\texttt{true}, s'', h''}
} \\
\drule{b.and.leftfalse}{
	\ang{B_1, s, h} \bsb \ang{\texttt{false}, s', h'} \\
	\ang{B_2, s', h'} \bsb \ang{\texttt{true}, s'', h''}
}{
	\ang{B_1 \ \& \ B_2, s, h} \bsb \ang{\texttt{false}, s'', h''}
} \\
\drule{b.and.rightfalse}{
	\ang{B_1, s, h} \bsb \ang{\texttt{true}, s', h'} \\
	\ang{B_2, s', h'} \bsb \ang{\texttt{false}, s'', h''}
}{
	\ang{B_1 \ \& \ B_2, s, h} \bsb \ang{\texttt{false}, s'', h''}
} \\
\drule{b.and.bothfalse}{
	\ang{B_1, s, h} \bsb \ang{\texttt{false}, s', h'} \\
	\ang{B_2, s', h'} \bsb \ang{\texttt{false}, s'', h''}
}{
	\ang{B_1 \ \& \ B_2, s, h} \bsb \ang{\texttt{false}, s'', h''}
} \\
\drule{b.not.true}{
	\ang{B, s, h} \bsb \ang{\texttt{false}, s', h'}
}{
	\ang{\lnot B, s, h} \bsb \ang{\texttt{true}, s', h'}
} \\
\drule{b.not.false}{
	\ang{B, s, h} \bsb \ang{\texttt{true}, s', h'}
}{
	\ang{\lnot B, s, h} \bsb \ang{\texttt{false}, s', h'}
}
\end{gather*}
\\ \\
\indent b) \\
\indent The expression (\newp) could cause \[\ang{E = E, s, h} \bse \ang{\texttt{false}, s', h'},\] as the memory address chosen for \newp is not constrained to a single choice. This can be proven using the derivation:
\begin{gather*}
\drule{b.eq.false}{
	(\dag) \\ 
	(\ddag) \\
	\ad{1} \neq \ad{3}
}{
	\ang{(\newp = \newp), s, h} \bsb \ang{\texttt{false}, s'', h''}
} \\
(\dag) \equiv \drule{e.new}{
		1 \notin \dom{h} \\ (1 + 1) \notin \dom{h}
	}{
		\ang{\newp, s, h} \bse \ang{\ad{1}, s, h[1 \mapsto 0, 2 \mapsto 0]}
	} \\
(\ddag) \equiv \drule{e.new}{
		3 \notin \dom{h} \\ (3 + 1) \notin \dom{h}
	}{
		\ang{\newp, s, h} \bse \ang{\ad{3}, s, h[3 \mapsto 0, 4 \mapsto 0]}
	}
\end{gather*}
\clearpage \noindent
\rule{\linewidth}{0.4pt} \\ \\
Question 3 \\
\indent a) \\
\begin{gather*}
\drule{c.seq}{
	(\dag) \\
	(\ddag)
}{
	\ang{x := \fst{c}; \texttt{fst}[c] \gets y; c := \snd{c}, s, h}
	\bsc
	\ang{s'', h'}
} \\ \\
(\dag) \equiv \drule{c.assign}{
		\drule{e.fst}{
			\drule{e.var}{
				c \in \dom{s}
			}{
				\ang{c, s, h} \bse \ang{\ad{5}, s, h}
			}
		}{
			\ang{\fst{c}, s, h} \bse \ang{3, s, h}
		}
	}{
		\ang{x := \fst{c}, s, h} \bsc \ang{s[x \mapsto 3], h}
	} \\ \\
(\ddag) \equiv \drule{c.seq}{
		(\ddag') \\
		(\ddag'')
	}{
		\ang{\texttt{fst}[c] \gets y; c := \snd{c}, s', h}
		\bsc
		\ang{s'', h'}
	} \\ \\
(\ddag') \equiv \drule{c.stor.fst}{
			\drule{e.var}{
				c \in \dom{s}
			}{
				\ang{c, s', h} \bse \ang{\ad{5}, s', h}
			} \\
			\drule{e.var}{
				y \in \dom{s}
			}{
				\ang{y, s', h} \bse \ang{12, s', h}
			} \\
			5 \in \dom{h}
		}{
			\ang{\texttt{fst}[c] \gets y, s', h}
			\bsc
			\ang{s', h[5 \mapsto 12]}
		} \\ \\
(\ddag'') \equiv \drule{c.assign}{
			\drule{e.snd}{
				\drule{e.var}{
					c \in \dom{s}
				}{
					\ang{c, s', h'} \bse \ang{\ad{5}, s', h'}
				}
			}{
				\ang{\snd{c}, s', h} \bse \ang{\ad{3}, s', h'}
			}
		}{
			\ang{c := \snd{c}, s', h'}
			\bsc
			\ang{s'[c \mapsto \ad{3}], h'}
		}
\end{gather*}
Where: \[s'  \equiv s[x \mapsto 3],\]
	  \[s'' \equiv s'[c \mapsto \ad{3}],\]
	  \[h'  \equiv h[5 \mapsto 12] \] \\
\indent b)
\indent \[\ang{C, s, h} \bsc \ang{s', h'}\] \\
Where: \[s' \equiv (hd \mapsto \ad{1}, c \mapsto 0, x \mapsto \ad{1}, y \mapsto 3) \]
	   \[h' \equiv (1 \mapsto 7, 2 \mapsto \ad{5}, 3 \mapsto 3, 4 \mapsto 0, 5 \mapsto 7, 6 \mapsto \ad{3})\] \\

\clearpage \noindent
\rule{\linewidth}{0.4pt} \\ \\
Question 4 \\
\indent a) \\
\indent i. $ \hptyp{h}{\ad{3}}{((nat, (nat, nat)), nat)} $ \\
\indent ii. There is no type $\typ$ such that $\hptyp{h}{\ad{9}}{\typ}$ \\
\indent iii. As the type $\typ_2$ (Where $\hptyp{h}{\ad{1 + 1}}{\typ_2}$) relies on the type $\typ$, where $\typ \equiv (\typ_1, \typ_2)$, there is a cycle on type reliance, so there is no type $\typ'$ such that $\hptyp{h}{\ad{1}}{\typ'}$ \\ \\
\indent b) \\
\indent i. $\tcompat{\Gamma}{s}{h}$, as both $\Gamma$ and $s \& h$ agree on the types of $x \& z$ \\
\indent ii. Here, it is not the case that $\tcompat{\Gamma}{s}{h}$, as $\etyp{\Gamma}{x}{(nat,(nat, nat))}$, but in $s;h$, $x \mapsto nat$ \\
\indent iii. Again, $\tcompat{\Gamma}{s}{h}$ \\
\indent c) SEE ATTACHED PAPER
\end{document}
