% LaTeX template for Models of Computation assessed coursework
% Most of the required packages are standard and should be provided by most TeX installations
% The exception is mathpartir, which may be downloaded from http://cristal.inria.fr/~remy/latex/

\documentclass[11pt,a4paper]{article}

\usepackage{fullpage}
\usepackage{rotating}
\usepackage{amsmath, amssymb, amsthm}
\usepackage{stmaryrd}
\usepackage{proof}
\usepackage{mathpartir}
\usepackage{tikz}
\usepackage{verbatim}
\usepackage{url}
\usepackage{multicol}
\usepackage{xr-hyper}

% Common macros

% BNF notation
\newcommand{\Gdef}{\mathrel{\mathop{::}}=}
\newcommand{\Gbar}{\mathbin{\ \big|\ }}
\newcommand{\Coloneqq}{\Gdef}

% Big-step arrow
\newcommand{\bigstep}{\mathrel{\Downarrow}}

% Semantic operators are (often) underlined to avoid ambiguity
\newcommand{\semop}[1]{\mathbin{\underline{#1}}}


% Program syntax is set in teletype using the \cmd macro
\newcommand{\cmd}[1]{\texttt{#1}}

% Macros for program constructs
\newcommand{\ifthen}[3]{
  \cmd{if} \; #1 \; \cmd{then} \; #2 \; \cmd{else} \; #3 }
\newcommand{\while}[2]{
  \cmd{while} \; #1 \; \cmd{do} \; #2 }

% \ang{x} typesets x in angled brackets
\newcommand{\ang}[1]{\langle #1 \rangle}

% The following two macros are for typesetting rules and derivations
% Usage: \drule{rule name}{premise1 \\ premise2 \\ premise3 ...}{conclusion}
% The premises will often also be derivations using \drule.
% The difference between \drule and and \Drule is that the space for the rule name
% is not measured with \Drule.  This is useful for typesetting left-most subderivations.
\newcommand{\drule}[3]{\inferrule*[left={#1}]{#2}{#3}}
\newcommand{\Drule}[3]{\inferrule*[Left={#1}]{#2}{#3}}

% For defininitions
\newcommand{\eqdef}{\triangleq}


% WhileDM
\newcommand{\whiledm}{\textsc{WhileDM}}
% Names of types
\newcommand{\tname}[1]{\mathit{#1}}
\newcommand{\exprs}{\tname{Expr}}
\newcommand{\bools}{\tname{Bool}}
\newcommand{\comms}{\tname{Comm}}
\newcommand{\vars}{\tname{Var}}
\newcommand{\nums}{\tname{Num}}
\newcommand{\addrs}{\tname{Addr}}
\newcommand{\vals}{\tname{Val}}
\newcommand{\stor}{\tname{Store}}
\newcommand{\heap}{\tname{Heap}}
\newcommand{\bv}{\tname{BVal}}
\newcommand{\ad}[1]{\ulcorner {#1} \urcorner}
\newcommand{\newp}{\texttt{newpair}}
\newcommand{\fst}[1]{{#1}.\texttt{fst}}
\newcommand{\snd}[1]{{#1}.\texttt{snd}}
\newcommand{\stof}[2]{\texttt{fst} [ {#1} ] \leftarrow {#2}}
\newcommand{\stos}[2]{\texttt{snd} [ {#1} ] \leftarrow {#2}}
\newcommand{\dom}[1]{\mathrm{dom}(#1)}
\newcommand{\bse}{\bigstep_e}
\newcommand{\bsc}{\bigstep_c}
\newcommand{\bsb}{\bigstep_b}
\newcommand{\types}{\tname{Type}}
\newcommand{\typ}{\tau} % Type variable
\newcommand{\tc}{\Gamma} % Type context
\newcommand{\tnat}{\mathsf{nat}}
\newcommand{\tpair}[2]{(#1,#2)}
\newcommand{\hptyp}[3]{#1 \Vdash #2 : #3}
\newcommand{\tcompat}[3]{#1 ; #2 ; #3 \vdash \textsf{\textup{well-typed}}}
\newcommand{\etyp}[3]{#1 \vdash #2 : #3}


\begin{document}
\title{Models of Computation Assessed Coursework I}
\author{A. Student}

\maketitle

This document is a skeleton which provides macros to help typeset your solution to the coursework.

Rules can be typeset like this:
% gather* is a math environment provided by the amsmath package which can typeset multiple lines of maths (separated by \\).
\begin{gather*}
  \drule{e.var}{ x \in \dom{s} }{\ang{x, s, h} \bse \ang{s(x), s, h}} \qquad
  \drule{e.num}{\ }{\ang{ n, s, h} \bse \ang{ n, s, h}} \\[0.5em]
  \drule{e.add}{
    \ang{E_1, s, h} \bse \ang{ n_1, s', h'} \\
    \ang{E_2, s', h'} \bse \ang{ n_2, s'', h''} \\
    n_3 = n_1 \semop+ n_2
  }{ \ang{E_1 + E_2, s, h} \bse \ang{n_3, s'', h''} } \\[0.5em]
  \drule{e.new}{ a \notin \dom h \\ (a + 1) \notin \dom h }{
    \ang{\newp, s, h} \bse \ang{\ad a, s, h[a \mapsto 0][a + 1 \mapsto 0]}
  } \displaybreak[1] \\[0.5em]
  \drule{e.fst}{ \ang{E, s, h} \bse \ang{\ad a, s', h'} \\ a \in \dom{h'} }{
    \ang{\fst{E}, s, h} \bse \ang{h'(a), s', h'}
  } \\[0.5em]
  \drule{e.snd}{ \ang{E, s, h} \bse \ang{\ad a, s', h'} \\ a+1 \in \dom{h'} }{
    \ang{\snd{E}, s, h} \bse \ang{h'(a+1), s', h'}
  }
\end{gather*}

Derivations can be typeset as in Figure~\ref{fig:deriveg}

\begin{sidewaysfigure}
% For very wide derivations, it can be helpful to write part of the derivation separately and use a symbol to stand for it, as in the following example:
\[
  (\dagger) \equiv \drule{$\lor$I-r}{
          \Drule{$\lnot$I}{
            \Drule{$\lnot$E}{
              \Drule{$\lor$I-l}{
                \lnot (p \lor \lnot p), p \vdash p
              }{
                \lnot (p \lor \lnot p), p \vdash p
              } \\
              \Drule{Ax}{\ }{\lnot (p \lor \lnot p) \vdash \lnot (p \lor \lnot p)}
            }{
              \lnot (p \lor \lnot p), p \vdash \bot
            }
          }{
            \lnot (p \lor \lnot p) \vdash \lnot p
          }
        }{
          \lnot (p \lor \lnot p) \vdash p \lor \lnot p
        }
\]
\[
  \drule{$\lnot\lnot$E}{
    \Drule{$\lnot$I}{
      \Drule{$\lnot$E}{
        (\dagger) \\
        \Drule{Ax}{\ }{\lnot (p \lor \lnot p) \vdash \lnot (p \lor \lnot p)}
      }{
        \lnot (p \lor \lnot p) \vdash \bot
      }
    }{
      \vdash \lnot \lnot (p \lor \lnot p)
    }
  }{
    \vdash p \lor \lnot p
  }
\]
\caption{A derivation}
\label{fig:deriveg}
\end{sidewaysfigure}

Types:
\begin{gather*}
  \tcompat{\Gamma}{s}{h} \\ % State is well-typed with respect to a context 
  \etyp{\Gamma}{E}{\typ} \\ % E has type \typ in context \Gamma
  \hptyp{h}{v}{\typ} \\ % Value v has type \typ in heap h
\end{gather*}

% For inline maths, use $ ... $
Suppose that $\tcompat{\Gamma}{s}{h}$.
% For display maths, use \[ \] (other options are available)
We wish to show that 
\[
  \etyp{\Gamma}{E}{\typ} \implies \hptyp{h}{v}{\typ} \text{.}
\]

\end{document}
